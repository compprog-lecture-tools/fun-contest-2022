\documentclass[
  a4paper,
  12pt,
  parskip=half,
  headings=standardclasses,
  footskip=0pt,
  footlines=1,
  headheight=80in
]{scrartcl}
\setlength{\textheight}{1.15\textheight}

\input{metainfo-include.tex}

\sloppy
\usepackage[utf8]{inputenc}
\usepackage{enumerate}
\usepackage{xcolor}
\usepackage{graphicx}
\usepackage{hyperref}
\usepackage{amsmath}
\usepackage{amssymb}
\usepackage{paracol}
\usepackage{palatino}
\usepackage{mathtools}
\usepackage{listings}
\usepackage[headsepline]{scrlayer-scrpage}
\usepackage[most, listings]{tcolorbox}
\usepackage{longtable}[=v4.13]
\usepackage{tabu}
\usepackage{upquote}
\usepackage{tikz}
\usetikzlibrary{shapes.misc, positioning}

\definecolor{darkblue}{rgb}{0,0,0.5}
\hypersetup{%
  colorlinks=true,
  urlcolor=darkblue
}

\pagestyle{scrheadings}
\setkomafont{pageheadfoot}{}
\ohead{\small CompProg Lecture, Tobias Friedrich (HPI)}
\ihead{\small ICPC Praktikum, Dorothea Wagner (KIT)}
\cfoot{}

\addtolength{\headheight}{-1.5em}
\newcommand{\makeheader}{%
  \columnratio{0.2,0.6,0.2}
  \begin{paracol}{3}
  \includegraphics[width=3cm]{kit-logo.pdf}
  \switchcolumn
  \center
  {\Large\textbf{Competitive Programming SS22}}\\
  \vspace{2mm}
  \textbf{Submit until \deadline{} via the \href{https://domjudge.iti.kit.edu/main/login}{judge}}
  \switchcolumn
  \hfill
  \includegraphics[width=3cm]{hpi-logo.pdf}\\
  \end{paracol}

  \noindent\textbf{Problem: \problemName{}} (\timelimit{} second timelimit) 
}

\newcommand{\placeholder}[1]{\textcolor{blue}{#1}}

\newenvironment{samples}[1][1]{%
  \noindent
  \begin{longtabu} to \textwidth {@{}X[#1] X[1]@{}}
    \textbf{Sample input} & \textbf{Sample output} \\
}{%
  \end{longtabu}
}
\newcommand{\sampleBox}[1]{%
  % Using frame=empty doesn't work for listings broken across pages
  \tcbinputlisting{listing file=#1, listing only, colframe=white, colback=black!10!white, sharp corners, box align=top}
}
\newcommand{\sample}[1]{%
  \sampleBox{../testcases/#1.in} & \sampleBox{../testcases/#1.ans} \\
}

\newcommand{\bonusnotice}{%
\textit{Note:} This is a bonus problem that is harder to solve than usual.
Solve the other problems first before spending too much time on this one.

\vspace{1em}
}


\begin{document}

\makeheader \ \\
After a long working day as the hero of TownyMcTownFace you go to your favorite restaurant "The Pannekoekhuis". Sadly when the waitress gives you the plate with the pannenkoeken you were longing for toadys hard work shows its downsides: Your arms not strong enough to hold the plate properly and it falls shattering into pieces. Not only your dreams of a tasty pannenkoeken shattered. The owner now wants you to repay the damage you caused by working in the kitchen. \\ \ \\
In the kitchen, the chef tells you to clean up the mess he created after he heard that you dropped one of her unbelievably tasteful pannenkoeken. You are standing in front of a pile of pannekoeken where some of the pannenkoeken are turned upside down. As the guests need their pannenkoeken served with the upside being correct, you now have to correct this mess thus all pannenkoeken are turned the correct way. The catch of your task is, that you only got one spatula and only one plate on which you are allowed to correct the mess. But using you inhuman swiftness of a hero you can always select $k$ top most pannenkoeken as a whole stack and turn this whole stack upside down and thereby correcting the turn of the pannenkoeken one after another. But remember, as your power of today is nearly depleted you can only do up to k many turns until your power runs out. You are very confident that your power is enough to solve this task. Can you show the chef your utter best performance of your secret pannenkoeken turning skills?


\paragraph*{Input}\ \\
The first line contains $n$ $(1 \leq n \leq  10^6)$, the number of pannenkoeken you need to sort followed by $k$ $(1 \leq k \leq  n)$, the number of pannenkoeken turns you can make. The next line contains the $n$-many string of either \emph{up} or \emph{down} separated by space given the orientation of each pannenkoeken starting from the top most pannenkoeken on the plate. \emph{up} means that the pannenkoeken is turned correctly and \emph{down} means that this \emph{down} needs to be turned.

\paragraph*{Output}\ \\
The first line should be an integer $\leq k$ giving the number of turns you want to make to sort the pannenkoekens. The next line should now contain $k$ many values where $t_i$ represents that for turn $i$ you turn the $t_i$ upper most pannenkoeken upside down.
\begin{samples}
  %\sample{sample1}
  %\sample{sample2}
\end{samples}

\end{document}