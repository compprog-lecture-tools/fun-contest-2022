\documentclass[
  a4paper,
  12pt,
  parskip=half,
  headings=standardclasses,
  footskip=0pt,
  footlines=1,
  headheight=80in
]{scrartcl}
\setlength{\textheight}{1.15\textheight}

\input{metainfo-include.tex}

\sloppy
\usepackage[utf8]{inputenc}
\usepackage{enumerate}
\usepackage{xcolor}
\usepackage{graphicx}
\usepackage{hyperref}
\usepackage{amsmath}
\usepackage{amssymb}
\usepackage{paracol}
\usepackage{palatino}
\usepackage{mathtools}
\usepackage{listings}
\usepackage[headsepline]{scrlayer-scrpage}
\usepackage[most, listings]{tcolorbox}
\usepackage{longtable}[=v4.13]
\usepackage{tabu}
\usepackage{upquote}
\usepackage{tikz}
\usetikzlibrary{shapes.misc, positioning}

\definecolor{darkblue}{rgb}{0,0,0.5}
\hypersetup{%
  colorlinks=true,
  urlcolor=darkblue
}

\pagestyle{scrheadings}
\setkomafont{pageheadfoot}{}
\ohead{\small CompProg Lecture, Tobias Friedrich (HPI)}
\ihead{\small ICPC Praktikum, Dorothea Wagner (KIT)}
\cfoot{}

\addtolength{\headheight}{-1.5em}
\newcommand{\makeheader}{%
  \columnratio{0.2,0.6,0.2}
  \begin{paracol}{3}
  \includegraphics[width=3cm]{kit-logo.pdf}
  \switchcolumn
  \center
  {\Large\textbf{Competitive Programming SS22}}\\
  \vspace{2mm}
  \textbf{Submit until \deadline{} via the \href{https://domjudge.iti.kit.edu/main/login}{judge}}
  \switchcolumn
  \hfill
  \includegraphics[width=3cm]{hpi-logo.pdf}\\
  \end{paracol}

  \noindent\textbf{Problem: \problemName{}} (\timelimit{} second timelimit) 
}

\newcommand{\placeholder}[1]{\textcolor{blue}{#1}}

\newenvironment{samples}[1][1]{%
  \noindent
  \begin{longtabu} to \textwidth {@{}X[#1] X[1]@{}}
    \textbf{Sample input} & \textbf{Sample output} \\
}{%
  \end{longtabu}
}
\newcommand{\sampleBox}[1]{%
  % Using frame=empty doesn't work for listings broken across pages
  \tcbinputlisting{listing file=#1, listing only, colframe=white, colback=black!10!white, sharp corners, box align=top}
}
\newcommand{\sample}[1]{%
  \sampleBox{../testcases/#1.in} & \sampleBox{../testcases/#1.ans} \\
}

\newcommand{\bonusnotice}{%
\textit{Note:} This is a bonus problem that is harder to solve than usual.
Solve the other problems first before spending too much time on this one.

\vspace{1em}
}


\usepackage{csquotes}

\begin{document}

\makeheader

Another contest, another design approach.
Last time, you had to make the best of the problems your TA's gave you, which you found awfully limiting.
This time around, you will select the difficulties upfront, and leave it to them to find matching problems.

You already decided that there should be $k$ problems with a difficulty sum of $n$.
There shouldn't be any two problems of the same difficulty, and all difficulties should be positive.
Your new approach is looking at the jump in difficulties from one problem to the next (assuming that they are solved in order of increasing difficulty).
For the easiest problem, you consider it as if the previous one had difficulty 0.

You assume that it is good if these jumps are \enquote{similar} by some metric.
As they can't always be exactly the same, you instead decided to look at their common GCD.\@
Can you design a difficulty progression that maximizes this GCD, or decide that is impossible to find any progression?

\paragraph*{Input}

The first line contains $t$ ($1 \leq t \leq 30$), the number of test cases.
Each test case consists of a line containing two numbers $n$ and $k$ ($1 \leq k \leq n \leq 10^9$), the desired difficulty sum and number of problems.

\paragraph*{Output}

For each test case, output a line.
If a valid progression exists, output $k$ integers $a_i$ ($1 \leq a_i \leq n$) in increasing order --- a progression maximizing the common GCD of the difficulty jumps.
Otherwise, output -1.

\begin{samples}
  \sample{sample}
\end{samples}

\paragraph*{Sample notes}

For the first case, \texttt{1 2 3} is the only valid progression since all others either have a larger sum or have duplicated difficulties.

For the second case, the difficulty jumps are $3$, $3$, and $6$, with a GCD of $3$.
There are multiple answers with this GCD, but it can be proven that any larger GCD is impossible.

For the third case, there are no four positive, unique numbers that sum to 7.

\end{document}
