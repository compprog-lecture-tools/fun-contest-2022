\input{template.tex}

\begin{document}

\makeheader

You are a medieval witch hunter hunting a notorious tribe of witches -- the “\underline{Hex}aflexagons”.
Since back then they had nothing, not even a second dimension, people, including witches, lived
on a perfect straight line segment between $-10^9$ and $10^9$. The distance is in feet (an old, obsolete unit,
no serious country would use anymore).

You start your hunt from your home at $h$. You know that the tribe's hex-address is $w$.
In your $i$-th step you can either add, subtract, multiply or integer divide your hex-address by $p_i$,
where $p_i$ is the $i$-th prime number.

Of course, you want to catch the evil witches as fast as possible, but
how many steps $s$ do you need to put an end to their wicked existence?
If you need more than $12$ steps, you can be assured that
the “\underline{Hex}aflexagons” have fled already.

But there's a catch:
RAM was quite expensive in the Middle Ages\textsuperscript{[\textcolor{blue}{Citation Needed}]},
so you can only afford to use 10 MB of RAM.

\paragraph*{Input}

The first line contains a integer $t$ ($1\leq t\leq 50$) the number of test cases.

Each test case consists of one line containing two integers%hexadecimal integers\footnote{You might want to use \url{https://en.cppreference.com/w/cpp/io/manip/hex} to read them.}
$h$ and $w$ ($-10^9 \leq h, w \leq 10^9$), the hex-address of your home and the hex-address of the witches, respectively, separated by a space.

It is guaranteed that the sum of steps $s$ over all test cases does not exceed $150$.

\paragraph*{Output}

Print a single integer, the minimum number of steps needed to catch the witches.
If you need more than $12$ steps to catch the witches, output $-1$.

\begin{samples}
  \sample{sample1}
\end{samples}

\end{document}