\documentclass[
  a4paper,
  12pt,
  parskip=half,
  headings=standardclasses,
  footskip=0pt,
  footlines=1,
  headheight=80in
]{scrartcl}
\setlength{\textheight}{1.15\textheight}

\input{metainfo-include.tex}

\sloppy
\usepackage[utf8]{inputenc}
\usepackage{enumerate}
\usepackage{xcolor}
\usepackage{graphicx}
\usepackage{hyperref}
\usepackage{amsmath}
\usepackage{amssymb}
\usepackage{paracol}
\usepackage{palatino}
\usepackage{mathtools}
\usepackage{listings}
\usepackage[headsepline]{scrlayer-scrpage}
\usepackage[most, listings]{tcolorbox}
\usepackage{longtable}[=v4.13]
\usepackage{tabu}
\usepackage{upquote}
\usepackage{tikz}
\usetikzlibrary{shapes.misc, positioning}

\definecolor{darkblue}{rgb}{0,0,0.5}
\hypersetup{%
  colorlinks=true,
  urlcolor=darkblue
}

\pagestyle{scrheadings}
\setkomafont{pageheadfoot}{}
\ohead{\small CompProg Lecture, Tobias Friedrich (HPI)}
\ihead{\small ICPC Praktikum, Dorothea Wagner (KIT)}
\cfoot{}

\addtolength{\headheight}{-1.5em}
\newcommand{\makeheader}{%
  \columnratio{0.2,0.6,0.2}
  \begin{paracol}{3}
  \includegraphics[width=3cm]{kit-logo.pdf}
  \switchcolumn
  \center
  {\Large\textbf{Competitive Programming SS22}}\\
  \vspace{2mm}
  \textbf{Submit until \deadline{} via the \href{https://domjudge.iti.kit.edu/main/login}{judge}}
  \switchcolumn
  \hfill
  \includegraphics[width=3cm]{hpi-logo.pdf}\\
  \end{paracol}

  \noindent\textbf{Problem: \problemName{}} (\timelimit{} second timelimit) 
}

\newcommand{\placeholder}[1]{\textcolor{blue}{#1}}

\newenvironment{samples}[1][1]{%
  \noindent
  \begin{longtabu} to \textwidth {@{}X[#1] X[1]@{}}
    \textbf{Sample input} & \textbf{Sample output} \\
}{%
  \end{longtabu}
}
\newcommand{\sampleBox}[1]{%
  % Using frame=empty doesn't work for listings broken across pages
  \tcbinputlisting{listing file=#1, listing only, colframe=white, colback=black!10!white, sharp corners, box align=top}
}
\newcommand{\sample}[1]{%
  \sampleBox{../testcases/#1.in} & \sampleBox{../testcases/#1.ans} \\
}

\newcommand{\bonusnotice}{%
\textit{Note:} This is a bonus problem that is harder to solve than usual.
Solve the other problems first before spending too much time on this one.

\vspace{1em}
}

\usepackage{hyperref}

\begin{document}

\makeheader

You are a medieval witch hunter hunting a notorious tribe of witches -- the “\underline{Hex}aflexagons”.
Since back then they had nothing, not even a second dimension, people, including witches, lived
on a perfect straight line segment between $0$ and $2 \cdot 10^9$. The distance is in feet (an old, obsolete unit,
no serious country would use anymore).

You start your hunt from your home at $h$. You know that the tribe's hex-address is $w$.
In your $i$-th step you can either add, subtract, multiply or integer divide your hex-address by $p_i$,
where $p_i$ is the $i$-th prime number.

Of course, you want to catch the evil witches as fast as possible, but
how many steps $s$ do you need to put an end to their wicked existence?
If you need more than $12$ steps, you can be assured that
the “\underline{Hex}aflexagons” have fled already.

But there's a catch:
RAM was quite expensive in the Middle Ages\textsuperscript{[\textcolor{blue}{Citation Needed}]},
so you can only afford to use 10 MB of RAM\footnote{If it happens that you need more, this may result in a \texttt{RUN-ERROR}.}.

\paragraph*{Input}

The first line contains a decimal integer $t$ ($1\leq t\leq 50$) the number of test cases.

Each test case consists of one line containing two hexadecimal integers\footnote{You might want to use \url{https://en.cppreference.com/w/cpp/io/manip/hex} or \texttt{int(input(), 16)} to read them.}
$h$ and $w$ ($0 \leq h, w \leq 2\cdot 10^9$), the hex-address of your home and the hex-address of the witches, respectively, separated by a space.

It is guaranteed that the sum of steps $s$ over all test cases does not exceed $150$.

\paragraph*{Output}

Print a single integer, the minimum number of steps needed to catch the witches.
If you need more than $12$ steps to catch the witches, output $-1$.

\begin{samples}
  \sample{sample1}
\end{samples}

\end{document}