\documentclass[
  a4paper,
  12pt,
  parskip=half,
  headings=standardclasses,
  footskip=0pt,
  footlines=1,
  headheight=80in
]{scrartcl}
\setlength{\textheight}{1.15\textheight}

\input{metainfo-include.tex}

\sloppy
\usepackage[utf8]{inputenc}
\usepackage{enumerate}
\usepackage{xcolor}
\usepackage{graphicx}
\usepackage{hyperref}
\usepackage{amsmath}
\usepackage{amssymb}
\usepackage{paracol}
\usepackage{palatino}
\usepackage{mathtools}
\usepackage{listings}
\usepackage[headsepline]{scrlayer-scrpage}
\usepackage[most, listings]{tcolorbox}
\usepackage{longtable}[=v4.13]
\usepackage{tabu}
\usepackage{upquote}
\usepackage{tikz}
\usetikzlibrary{shapes.misc, positioning}

\definecolor{darkblue}{rgb}{0,0,0.5}
\hypersetup{%
  colorlinks=true,
  urlcolor=darkblue
}

\pagestyle{scrheadings}
\setkomafont{pageheadfoot}{}
\ohead{\small CompProg Lecture, Tobias Friedrich (HPI)}
\ihead{\small ICPC Praktikum, Dorothea Wagner (KIT)}
\cfoot{}

\addtolength{\headheight}{-1.5em}
\newcommand{\makeheader}{%
  \columnratio{0.2,0.6,0.2}
  \begin{paracol}{3}
  \includegraphics[width=3cm]{kit-logo.pdf}
  \switchcolumn
  \center
  {\Large\textbf{Competitive Programming SS22}}\\
  \vspace{2mm}
  \textbf{Submit until \deadline{} via the \href{https://domjudge.iti.kit.edu/main/login}{judge}}
  \switchcolumn
  \hfill
  \includegraphics[width=3cm]{hpi-logo.pdf}\\
  \end{paracol}

  \noindent\textbf{Problem: \problemName{}} (\timelimit{} second timelimit) 
}

\newcommand{\placeholder}[1]{\textcolor{blue}{#1}}

\newenvironment{samples}[1][1]{%
  \noindent
  \begin{longtabu} to \textwidth {@{}X[#1] X[1]@{}}
    \textbf{Sample input} & \textbf{Sample output} \\
}{%
  \end{longtabu}
}
\newcommand{\sampleBox}[1]{%
  % Using frame=empty doesn't work for listings broken across pages
  \tcbinputlisting{listing file=#1, listing only, colframe=white, colback=black!10!white, sharp corners, box align=top}
}
\newcommand{\sample}[1]{%
  \sampleBox{../testcases/#1.in} & \sampleBox{../testcases/#1.ans} \\
}

\newcommand{\bonusnotice}{%
\textit{Note:} This is a bonus problem that is harder to solve than usual.
Solve the other problems first before spending too much time on this one.

\vspace{1em}
}

\usepackage{tkz-euclide}

\begin{document}

\makeheader

Lea recently learned about convex hulls and wanted to
do some more experiments herself.
Therefore, she nailed $n$ nails into a plank of wood and
pulled a rubber band over the nails.
Since Lea nailed nailing the nails, all nails are on
pairwise distinct spots.
But, unfortunately, the rubber band wasn't Convex Hull™ proof!
Instead, it snaps towards its center as illustrated in the figure down below.
\begin{figure}[ht]
  \centering
  \begin{tikzpicture}[scale=.75]
    \tkzInit[xmax=4,ymax=3,xmin=-3,ymin=-5]
    \tkzGrid
    \tkzAxeXY
    \draw[red, line width=2pt] (2,0) -- (2,-1) -- (4,-1) -- (3,-1)
    -- (3,-3) --(2,-3) -- (2,-4) -- (-1,-4)
    -- (-1,-5) -- (-1,-3) -- (-2,-3) -- (-2,-1)
    -- (-3,-1) -- (-2,-1) -- (-2,3) -- (-2,2)
    -- (1,2) -- (1,0) -- cycle;
    \draw[red, line width=4pt, transform canvas={yshift=-2pt}] (3,-1) -- (4,-1);
    \draw[red, line width=4pt, transform canvas={xshift=2pt}] (-1,-4) -- (-1,-5);
    \draw[red, line width=4pt] (-2,-1) -- (-3,-1);
    \draw[red, line width=4pt, transform canvas={xshift=2pt}] (-2,2) -- (-2,3);
    \foreach \point in {(1,2),(1,-2),(-1,-5),(-2,-3),(-2,3),(4,-1),(3,-3),(2,-4),(2,0),(-3,-1),(0,0)} {
        \fill[blue] \point circle[radius=3pt];
      }
  \end{tikzpicture}
  \caption{Sample \#2. Note: the rubber band sometimes uses some line segments twice (tick lines).}
  \label{fig:convexhull}
\end{figure}
Anyways, Lea still wants to know the length of the rubber band.
Can you help her?

\paragraph*{Input}

The first line contains an integer $n$ ($3 \leq n \leq 2\cdot10^5$) the number of nails.
Then $n$ lines follow, each containing two integers $x_i$ and $y_i$ ($-10^9 \leq x_i, y_i \leq 10^9$)
the coordinates of the $i$-th nail.

\paragraph*{Output}

Print a single integer denoting the length of the rubber band.

\begin{samples}
  \sample{sample1}
  \sample{sample2}
\end{samples}

\end{document}