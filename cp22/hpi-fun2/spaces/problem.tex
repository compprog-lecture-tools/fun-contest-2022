\documentclass[
  a4paper,
  12pt,
  parskip=half,
  headings=standardclasses,
  footskip=0pt,
  footlines=1,
  headheight=80in
]{scrartcl}
\setlength{\textheight}{1.15\textheight}

\input{metainfo-include.tex}

\sloppy
\usepackage[utf8]{inputenc}
\usepackage{enumerate}
\usepackage{xcolor}
\usepackage{graphicx}
\usepackage{hyperref}
\usepackage{amsmath}
\usepackage{amssymb}
\usepackage{paracol}
\usepackage{palatino}
\usepackage{mathtools}
\usepackage{listings}
\usepackage[headsepline]{scrlayer-scrpage}
\usepackage[most, listings]{tcolorbox}
\usepackage{longtable}[=v4.13]
\usepackage{tabu}
\usepackage{upquote}
\usepackage{tikz}
\usetikzlibrary{shapes.misc, positioning}

\definecolor{darkblue}{rgb}{0,0,0.5}
\hypersetup{%
  colorlinks=true,
  urlcolor=darkblue
}

\pagestyle{scrheadings}
\setkomafont{pageheadfoot}{}
\ohead{\small CompProg Lecture, Tobias Friedrich (HPI)}
\ihead{\small ICPC Praktikum, Dorothea Wagner (KIT)}
\cfoot{}

\addtolength{\headheight}{-1.5em}
\newcommand{\makeheader}{%
  \columnratio{0.2,0.6,0.2}
  \begin{paracol}{3}
  \includegraphics[width=3cm]{kit-logo.pdf}
  \switchcolumn
  \center
  {\Large\textbf{Competitive Programming SS22}}\\
  \vspace{2mm}
  \textbf{Submit until \deadline{} via the \href{https://domjudge.iti.kit.edu/main/login}{judge}}
  \switchcolumn
  \hfill
  \includegraphics[width=3cm]{hpi-logo.pdf}\\
  \end{paracol}

  \noindent\textbf{Problem: \problemName{}} (\timelimit{} second timelimit) 
}

\newcommand{\placeholder}[1]{\textcolor{blue}{#1}}

\newenvironment{samples}[1][1]{%
  \noindent
  \begin{longtabu} to \textwidth {@{}X[#1] X[1]@{}}
    \textbf{Sample input} & \textbf{Sample output} \\
}{%
  \end{longtabu}
}
\newcommand{\sampleBox}[1]{%
  % Using frame=empty doesn't work for listings broken across pages
  \tcbinputlisting{listing file=#1, listing only, colframe=white, colback=black!10!white, sharp corners, box align=top}
}
\newcommand{\sample}[1]{%
  \sampleBox{../testcases/#1.in} & \sampleBox{../testcases/#1.ans} \\
}

\newcommand{\bonusnotice}{%
\textit{Note:} This is a bonus problem that is harder to solve than usual.
Solve the other problems first before spending too much time on this one.

\vspace{1em}
}


\begin{document}

\makeheader

You get a late night call from your best buddy Bill. He asks you if you can help him with his problem.
Bill is the founder of a software company which sells a word processing program. He needs a program, which places words minimally ragged on lines with a fixed width.
A text is considered minimally ragged, if the squared sum of the remaining spaces on a line is minimal.

\paragraph*{Input}
A number $n$ ($1 \leq n \leq 10^6$), the number of words to place on the lines.
A list of $n$ words on a single line, each word has at most 20 characters. 
A line width $w$ ($1 \leq w \leq 250$), defining the maxiumum number of characters on a single line.
You can assume that every word fits on a line.

\paragraph*{Sample explanation}
For example we have the words $aaa$ $bb$ $ccc$, and a line width of 6.
A minimal placement would place the words $aaa$ and $bb$ on one line. In this line there is no remaining space.
In the next line the word $ccc$ is placed. Because this is the last line the space is not counted.
So the minimum cost is: $0^2 = 0$.
Lets consider another example: $aaa$ $bbb$ $ccc$. Here every word must be placed in his own line.
So the minumum cost is: $3^2+3^2= 27$.

\paragraph*{Output}

Print a single integer - the minimum cost to place all words.

\begin{samples}
  \sample{sample1}
  \sample{sample2}
\end{samples}

\end{document}