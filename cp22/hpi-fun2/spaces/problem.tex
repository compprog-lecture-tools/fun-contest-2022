\input{template.tex}

\begin{document}

\makeheader

You get a late night call from your best buddy Bill. He asks you if you can help him with his problem.
Bill is the founder of a software company which sells a word processing program. He needs a program, which places words minimally ragged on lines with a fixed width.
A text is considered minimally ragged, if the squared sum of the remaining spaces on a line is minimal.

\paragraph*{Input}
A number $n$ ($1 \leq n \leq 10^6$), the number of words to place on the lines.
A list of $n$ words on a single line, each word has at most 20 characters. 
A line width $w$ ($1 \leq w \leq 250$), defining the maxiumum number of characters on a single line.
You can assume that every word fits on a line.

\paragraph*{Sample explanation}
For example we have the words $aaa$ $bb$ $ccc$, and a line width of 6.
A minimal placement would place the words $aaa$ and $bb$ on one line. In this line there is no remaining space.
In the next line the word $ccc$ is placed. Because this is the last line the space is not counted.
So the minimum cost is: $0^2 = 0$.
Lets consider another example: $aaa$ $bbb$ $ccc$. Here every word must be placed in his own line.
So the minumum cost is: $3^2+3^2= 27$.

\paragraph*{Output}

Print a single integer - the minimum cost to place all words.

\begin{samples}
  \sample{sample1}
  \sample{sample2}
\end{samples}

\end{document}